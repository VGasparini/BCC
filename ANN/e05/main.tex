\documentclass{article}
\usepackage[utf8]{inputenc}
\usepackage{amsmath}
\usepackage{listings}
\newcommand\tab[1][0.5cm]{\hspace*{#1}}

\title{E04 - ANN}
\author{Vinicius Gasparini}
\date{17 de Setembro de 2019}

\begin{document}

\maketitle

\section{Jacobi - Implementação}
\begin{lstlisting}[frame=single]
from pprint import pprint
from numpy import array, diag, diagflat, dot, linalg

def jacobi(A,b,N,x):                                                                                                                                                          
    D_ = diagflat(diag(A))
    L_U  = A - D_
    D_ = linalg.inv(D_)
                                                                                                                                                                         
    for i in range(N-1):
        x = dot(dot(-D_,L_U) , x)+dot(D_,b)
        print('X(%d) = '%(i+1),end='')
        print(x)
    return x

A = array([ [ 7.6 ,-0.2 ,-0.6 , 2   ,-1.3 , 1.2 , 2.3 ],
            [-2.4 , 18  ,-0.9 , 2.7 , 2.1 ,-2   , 0.4],
            [ 0.3 , 0.9 ,16.8 ,-1.9 , 1.3 , 2.2 ,-1.5],
            [-2   ,-1.2 , 2.4 ,11.8 ,-1.6 ,-1.6 ,-1.1],
            [-1.8 ,-2.5 , 2.4 ,-1.4 ,13.1 , 1.7 ,-2.7],
            [-2.2 , 1.8 , 1.9 ,-0.6 ,-0.1 , 10  , 1.3],
            [-1.5 ,-2   , 2.5 , 1.8 , 0.7 , 1.2 , 12.1]])

b = array([-0.8,-1.6,-4.6,0.8,0.8,4.6,4.0])
chute = array([-4.9,1.8,3.8,-2.8,-2.6,0.2,-2])

sol = jacobi(A,b,N=10,x=chute)
\end{lstlisting}
\newpage
\section{Resposta}
\begin{lstlisting}[frame=single]
A: array([[ 7.6, -0.2, -0.6,  2. , -1.3,  1.2,  2.3],
       [-2.4, 18. , -0.9,  2.7,  2.1, -2. ,  0.4],
       [ 0.3,  0.9, 16.8, -1.9,  1.3,  2.2, -1.5],
       [-2. , -1.2,  2.4, 11.8, -1.6, -1.6, -1.1],
       [-1.8, -2.5,  2.4, -1.4, 13.1,  1.7, -2.7],
       [-2.2,  1.8,  1.9, -0.6, -0.1, 10. ,  1.3],
       [-1.5, -2. ,  2.5,  1.8,  0.7,  1.2, 12.1]])

b: [-0.8 -1.6 -4.6  0.8  0.8  4.6  4. ]

x: [-0.30957065 -0.13108813 -0.28340396
    0.15146424  0.05709781  0.44505971
    0.25836421]
\end{lstlisting}
Pontando, a resposta correta é o \textit{\textbf{item c}}
\end{document}

